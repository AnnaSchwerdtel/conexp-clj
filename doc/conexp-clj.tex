\documentclass{scrbook}
\usepackage[T1]{fontenc}
\usepackage[utf8x]{inputenc}
\usepackage[english]{babel}

\usepackage{graphicx}
\usepackage{color}
\definecolor{lightblue}{rgb}{0.2,0.2,1}

\usepackage{hyperref}
\usepackage{enumerate}
\usepackage{array}
\usepackage{epigraph}

%%%

\newcommand{\conexpclj}{\texttt{conexp-clj}}
\newcommand{\set}[1]{\{\,#1\,\}}

\usepackage{listings}
\lstnewenvironment
  {conexp}
  {\lstset{
      language=Lisp,
      escapechar=§,
      basicstyle=\ttfamily,
      columns=fullflexible,
      % numbers
      numbers=left,
      stepnumber=5,
      numberstyle=\tiny,
      firstnumber=1,}}
  {}
\newcommand{\s}{\color{blue}}
\newcommand{\p}{\color{lightblue}}

\usepackage{amsmath,latexsym,amssymb}
\usepackage[amsmath,thmmarks,thref]{ntheorem}

\theoremstyle{plain}
\theoremheaderfont{\normalfont\bfseries}\theorembodyfont{\slshape}
\theoremsymbol{\ensuremath{\square}}
\newtheorem{Theorem}          {Theorem} [section]
\newtheorem{Proposition} [Theorem] {Proposition}
\newtheorem{Lemma}       [Theorem] {Lemma}
\newtheorem{Corollary}   [Theorem] {Corollary}

\theoremstyle{plain}
\theoremheaderfont{\normalfont}\theorembodyfont{\normalfont}
\theoremsymbol{}
\newtheorem{Remark}      [Theorem] {Remark}
\theoremsymbol{\ensuremath{\diamondsuit}}
\newtheorem{Example}     [Theorem] {Example}

\theoremstyle{plain}
\theoremheaderfont{\normalfont\bfseries}\theorembodyfont{\normalfont}
\theoremsymbol{\ensuremath{\diamondsuit}}
\newtheorem{Definition}  [Theorem] {Definition}

\theoremstyle{nonumberplain}
\theoremheaderfont{\normalfont\itshape}\theorembodyfont{\normalfont}
\theoremsymbol{\ensuremath{\square}}
\newtheorem{Proof}                 {Proof}


\title{\conexpclj\\A ConExp Rewrite in Clojure\\[2cm]DRAFT VERSION}
\author{DB}
\date{\today}


\begin{document}

\maketitle


\chapter*{Preface}

This is an attempt to reimplement some functionalities of ConExp in a
new programm called \conexpclj.


\chapter*{Acknowledgements}

John McCarthy for inventing Lisp, Richard Stallman for inventing Free
Software and Rich Hickey for inventing Clojure.


\tableofcontents


\chapter{Introduction}

\section{Using \conexpclj}

\subsection{A Sample Session}

\conexpclj\ can be run from a standard console as can be found on
every UNIX system. Simply issue the command
\begin{conexp}
  conexp-clj.sh
\end{conexp}
and you will see something like
\begin{conexp}
  Clojure 1.2.0-master-SNAPSHOT
  §\p conexp=>§
\end{conexp}
This is a Clojure-\emph{REPL} (Read-Eval-Print-Loop) which gives us
access to all functions available in \conexpclj. Let's have a look at
a sample session.

\begin{Example}~
  \begin{conexp}
    §\p conexp=>§ (§\s def§ ctx (§\s make-context§ #{1 2 3} #{1 2 3} #{[1 2] [2 2] [3 3]}))
    #'user/ctx
    §\p conexp=>§ ctx
    #<Context
      |1 2 3
    --+------
    1 |. x .
    2 |. x .
    3 |. . x
    >
    §\p conexp=>§ (§\s stem-base§ ctx)
    #{#<Implication ( #{1}  ==>  #{2 3} )> #<Implication ( #{2 3}  ==>  #{1} )>}
    §\p conexp=>§ (§\s reduced?§ ctx)
    false
    §\p conexp=>§ (§\s reduce-context§ ctx)
    #<Context
      |2 3
    --+----
    1 |x .
    3 |. x
    >
    §\p conexp=>§
  \end{conexp}

  It is quite obvious what we are doing here: First we define the
  context $(\set{1,2,3},\set{1,2,3},\break\set{(1,2),(2,2),(3,3)})$,
  then compute its stem base, find out that this context is not
  reduced (which is no surprise) and reduce it. This can all be
  guessed but the actual syntax looks quite strange and very
  unfamiliar to those who never have seen some Lisp before (or have
  and did not like it). But this syntax is very simple and easy to
  learn, so let us have some closer look at it.
\end{Example}

\subsection{Some Basic Lisp}

\section{Obtaining and Installing \conexpclj}

See \url{http://www.math.tu-dresden.de/~borch/conexp-clj}.


\section{Design Principles}

\subsection{Use Lisp}
The choice of a Lisp dialect as implementation language is not by
coincidence. Lisp is and has always been a language perfectly suitable
for both rapid development and thorough implementations. As this it is
both optimal as an implementation for the implementor and an interface
language for the user.

This decision has impacts on the users as she will need to work with
an unfamiliar looking programming language (if she has not been
working with Lisp before).

...

\subsection{Executable Math}

\subsection{GUI is optional}

\subsection{Easy Extensibility}


\section{Plans}

\subsection{Todo}

\subsection{What is missing but needed urgently?}

\begin{itemize}
\item A working GUI
\end{itemize}

%%%
\chapter{How Do I ...}

\section{... the concepts of a context?}
\section{... explore the attributes of a context?}
\section{... compute the valid implications of a context?}
\section{... compute and draw a concept lattice?}
\section{... compute the Luxenburger Basis?}

%%%
\chapter{Program Overview}

\appendix

\begin{thebibliography}{xxxxxxx}
\bibitem[GW]{GW} Ganter Wille
\end{thebibliography}

\end{document}
