\section{conexp.base}
\begin{description}
  \item[<=>]
([a b])

Implements equivalence.

  \item[=>]
([a b])

Implements implication.

  \item[abs]
([n])

(abs n) is the absolute value of n

  \item[all-closed-sets]
([G clop])

Computes all closed sets of a given closure operator on a given set.

  \item[ceil]
([n])

(ceil n) returns the least integer greater than or equal to n.
If n is an exact number, ceil returns an integer, otherwise a double.

  \item[compile-conexp]
([])

Compiles Java classes needed for conexp-clj.

  \item[cross-product]
([\& sets])

Returns cross product of set-1 and set-2.

  \item[crossfoot]
([n])

Returns the crossfoot of n.

  \item[difference]
([s1] [s1 s2] [s1 s2 \& sets])

Return a set that is the first set without elements of the remaining sets

  \item[disjoint-union]
([\& sets])

Computes the disjoint union of sets by joining the cross-products of the
  sets with natural numbers.

  \item[distinct-by-key]
([sequence key])

Returns a sequence of all elements of the given sequence with distinct key values,
  where key is a function from the elements of the given sequence. If two elements
  correspond to the same key, the one is chosen which appeared earlier in the sequence.

  This function is copied from clojure.core/distinct and adapted for using a key function.

  \item[ensure-length]
([string length] [string length padding])

Fills given string with padding to have at least the given length.

  \item[exact-integer-sqrt]
([n])

(exact-integer-sqrt n) expects a non-negative integer n, and returns [s r] where n = s\verb+^+2+r and n < (s+1)\verb+^+2.  In other words, it returns the floor of the square root and the 'remainder'.
For example, (exact-integer-sqrt 15) is [3 6] because 15 = 3\verb+^+2+6.

  \item[exists]
([bindings condition])

Implements logical exists quantor.

  \item[expt]
([base pow])

(expt base pow) is base to the pow power.
Returns an exact number if the base is an exact number and the power is an integer, otherwise returns a double.

  \item[factorial]
([n])

Returns n!.

  \item[fibs]
([])

Returns a lazy sequence of all the Fibonacci numbers.

  \item[first-non-nil]
([seq])

Returns first non-nil element in seq.

  \item[flatten]
([in])

Flattens in to be a sequence of depth 1 at most.

  \item[floor]
([n])

(floor n) returns the greatest integer less than or equal to n.
If n is an exact number, floor returns an integer, otherwise a double.

  \item[forall]
([bindings condition])

Implements logical forall quantor.

  \item[gcd]
([a b])

(gcd a b) returns the greatest common divisor of a and b

  \item[graph-of-function?]
([relation source target])

Returns true iff relation is the graph of a function from source to target.

  \item[illegal-argument]
([\& strings])

Throws IllegalArgumentException with given strings as message

  \item[index]
([xrel ks])

Returns a map of the distinct values of ks in the xrel mapped to a
  set of the maps in xrel with the corresponding values of ks.

  \item[intersection]
([s1] [s1 s2] [s1 s2 \& sets])

Return a set that is the intersection of the input sets

  \item[join]
([xrel yrel] [xrel yrel km])

When passed 2 rels, returns the rel corresponding to the natural
  join. When passed an additional keymap, joins on the corresponding
  keys.

  \item[lcm]
([a b])

(lcm a b) returns the least common multiple of a and b

  \item[lectic-<]
([G A B])

Implements lectic ordering. The basic order is given by the ordering of G
  which is interpreted as increasing order.

  \item[lectic-<\_i]
([G i A B])

Implements lectic < at position i. The basic order is given by the ordering
  of G which is interpreted as increasing order.

  \item[map-invert]
([m])

Returns the map with the vals mapped to the keys.

  \item[math-type]
([thing])

Dispatch function for multimethods. Identifies sets and sequences
  as :conexp.util/set and functions as :conexp.util/fn, all other as
  :conexp.util/other.

  \item[memo-fn]
([name args \& body])



  \item[next-closed-set]
([G clop A])

Computes next closed set with the Next Closure Algorithm. The order of elements in G,
  interpreted as increasing, is taken to be the basic order of the elements.

  \item[oplus]
([G clop A i])

Implements oplus of the Next Closure Algorithm.

  \item[powers-of-2]
([])

Returns a lazy sequence of all the powers of 2

  \item[prime?]
([n])

Returns true iff n is prime with a certainty of (- 1 (/ 1 (expt 2 1000)))

  \item[primes]


Lazy sequence of all the prime numbers.

  \item[project]
([xrel ks])

Returns a rel of the elements of xrel with only the keys in ks

  \item[proper-subset?]
([set1 set2])

Is s1 a proper subset of s2?

  \item[proper-superset?]
([set1 set2])

Is s1 a proper superset of s2?

  \item[recur-sequence]
([\& things])



  \item[rename]
([xrel kmap])

Returns a rel of the maps in xrel with the keys in kmap renamed to the vals in kmap

  \item[rename-keys]
([map kmap])

Returns the map with the keys in kmap renamed to the vals in kmap

  \item[round]
([n])

(round n) rounds to the nearest integer.
round always returns an integer.  Rounds up for values exactly in between two integers.

  \item[select]
([pred xset])

Returns a set of the elements for which pred is true

  \item[set-of]
([thing condition])

Macro for writing sets as mathematicians do (at least similar to it.)

  \item[set-of-range]
([end] [start end] [start end step])

Returns a set of all numbers from start upto (but not including) end,
  by step if provided.

  \item[split-at-first]
([predicate sequence])

Splits given sequence at first element satisfing predicate.
  The first element satisfing predicate will be in the second sequence.

  \item[split-at-last]
([predicate sequence])

Splits given sequence at last element satisfing predicate.
  The last element satisfing predicate will be in the first sequence.

  \item[sqrt]
([n])

Square root, but returns exact number if possible.

  \item[subelts]
([seq i])

Returns a subsequence of seq up to index i.

  \item[subset?]
([set1 set2])

Is set1 a subset of set2?

  \item[subsets]
([set])

Returns all subsets of set.

  \item[superset?]
([set1 set2])

Is set1 a superset of set2?

  \item[transitive-closure]
([set-of-pairs] [set-of-pairs new old])

Computes transitive closure of a given set of pairs.

  \item[union]
([] [s1] [s1 s2] [s1 s2 \& sets])

Return a set that is the union of the input sets

  \item[with-profiled-fns]
([fns \& body])



  \item[with-recur-seqs]
([seq-definitions \& body])



  \item[with-str-out]
([\& body])

Returns string of all output being made in (flatten body).

  \item[zip]
([seq-1 seq-2])

Returns sequence of pairs [x,y] where x runs through seq-1 and
  y runs through seq-2 simultaneously. This is the same as
  (map \#(vector \%1 \%2) seq-1 seq-2).

\end{description}

\section{conexp.fca}
\begin{description}
  \item[add-immediate-elements]
([implications initial-set])



  \item[add-immediate-elements*]
([implications initial-set])



  \item[adiag-context]
([base-set])



  \item[ask]
([prompt pred fail-message])



  \item[attribute-clarified?]
([ctx])



  \item[attribute-derivation]
([ctx attributes])



  \item[attributes]


Returns the attributes of a formal context.

  \item[base-set]
([lattice])



  \item[biordinal-scale]
([base n] [base n <= >=])

base must be ordered (e.g. vector or list). Otherwise the result will be arbitrary.

  \item[clarified?]
([ctx])



  \item[clarify-attributes]
([ctx])



  \item[clarify-context]
([ctx])



  \item[clarify-objects]
([ctx])



  \item[clop-by-implications]
([implications])



  \item[clop-by-implications*]
([implications])



  \item[close-under-implications]
([implications set])



  \item[compare-order]
([x y])

Orders things for proper output of formal contexts.

  \item[concept-lattice]
([ctx])



  \item[concept?]
([ctx [set-of-obj set-of-att]])



  \item[concepts]
([ctx])



  \item[conclusion]
([impl])



  \item[context-apposition]
([ctx-1 ctx-2])



  \item[context-attribute-closure]
([ctx set-of-attributes])



  \item[context-composition]
([ctx-1 ctx-2])



  \item[context-extends]
([ctx])



  \item[context-intents]
([ctx])



  \item[context-intersection]
([ctx1 ctx2])



  \item[context-object-closure]
([ctx set-of-objects])



  \item[context-product]
([ctx-1 ctx-2])



  \item[context-semiproduct]
([ctx-1 ctx-2])



  \item[context-subposition]
([ctx-1 ctx-2])



  \item[context-sum]
([ctx-1 ctx-2])



  \item[context-transitive-closure]
([ctx])



  \item[context-union]
([ctx1 ctx2])



  \item[context-xia-product]
([ctx-1 ctx-2])



  \item[default-handler]
([ctx impl])



  \item[diag-context]
([base-set])



  \item[dichotomic-scale]
([base])



  \item[directly-neighboured?]
([lat x y])



  \item[distributive?]
([lat])



  \item[down-arrows]
([ctx])



  \item[dual-context]
([ctx])



  \item[dual-lattice]
([lat])



  \item[explore-attributes]
([ctx handler])



  \item[falsifies-implication?]
([new-atts impl])



  \item[follows-semantically?]
([implication implications])



  \item[holds?]
([impl ctx])



  \item[incidence]


Returns the incidence of a formal context.

  \item[inf]
([lattice])



  \item[interordinal-scale]
([base] [base <= >=])



  \item[invert-context]
([ctx])



  \item[lattice-atoms]
([lat])



  \item[lattice-coatoms]
([lat])



  \item[lattice-inf-irreducibles]
([lat])



  \item[lattice-irreducibles]
([lat])



  \item[lattice-lower-neighbours]
([lat y])



  \item[lattice-one]
([lat])



  \item[lattice-sup-irreducibles]
([lat])



  \item[lattice-upper-neighbours]
([lat x])



  \item[lattice-zero]
([lat])



  \item[make-context]


Standard constructor for contexts. Takes a sequence of objects,
  a sequence of attributes and either a set of pairs or function of
  two elements being true iff its arguments are incident. Note that the
  object and attribute sequences are converted to sets and therefore have to
  not contain any douplicate elements.

  \item[make-implication]
([premise conclusion])

Creates an implication (premise => conclusion $\setminus$ premise).

  \item[make-lattice]




  \item[make-mv-context]




  \item[modular?]
([lat])



  \item[nominal-scale]
([base])



  \item[null-context]
([base-set])



  \item[object-clarified?]
([ctx])



  \item[object-derivation]
([ctx objects])



  \item[objects]


Returns the objects of a formal context.

  \item[one-context]
([base-set])



  \item[order]
([lattice])



  \item[ordinal-scale]
([base] [base <=])



  \item[premise]
([impl])



  \item[print-context]
([ctx order-on-objects order-on-attributes])

Prints contexts in a human readable form.

  \item[print-mv-context]
([mv-ctx])



  \item[reduce-clarified-context]
([ctx])



  \item[reduce-context]
([ctx])



  \item[reduced?]
([ctx])



  \item[rename-attributes]
([ctx old-to-new])

Rename attributes in ctx by given hash old-to-new.

  \item[rename-objects]
([ctx old-to-new])

Rename objects in ctx by given hash old-to-new.

  \item[respects?]
([set impl])



  \item[scale-mv-context]
([mv-ctx scales])



  \item[sort-by-first]
([x y])

Convenience function for custom context printing.

  \item[sort-by-second]
([x y])

Ensures that pairs are ordered by second entry first. This gives
  better output for context sums, products, ...

  \item[standard-context]
([lat])



  \item[stem-base]
([ctx])



  \item[sup]
([lattice])



  \item[type-of]
([thing])



  \item[up-arrows]
([ctx])



  \item[up-down-arrows]
([ctx])



\end{description}

\section{conexp.io}
\begin{description}
  \item[add-context-input-format]
([name predicate])



  \item[find-context-input-format]
([file])



  \item[get-known-context-input-formats]
([])



  \item[read-context]




  \item[write-context]




\end{description}

\section{conexp.gui}
\begin{description}
  \item[conexp-gui]
([])



  \item[show-in-frame]




\end{description}

\section{conexp.layout}
\begin{description}
  \item[draw-lattice]
([lattice layout])

Draws given lattice with given layout on a canvas and returns it.

  \item[lattice->graph]
([lattice])

Converts given lattice to it's corresponding graph with loops
  removed.

  \item[layer-coordinates]
([number layer])

Assigns coordinates to a given layer such that it is centerer
  around 0 at height given by number.

  \item[layers]
([lattice])

Returns the layers of the given lattice, that is sequence of points
  with equal heights.

  \item[scale-layout]
([[x1 y1] [x2 y2] [points-to-coordinates point-connections]])

Scales given layout to rectangle (x1 y1), (x2 y2). Layout is given
  as a map of points to coordinates and a sequence of connection pairs.

  \item[scale-points-to-rectangle]
([[x1 y1] [x2 y2] points])

Scales the collection of points such that they fit in the
  rectangle given by [x1 y1] and [x2 y2].

  \item[simple-layered-layout]
([lattice])

Simple layered layout for lattice visualization.

\end{description}

